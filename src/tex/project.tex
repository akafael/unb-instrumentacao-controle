% Pacotes e configurações padrão do estilo ``article''\
% -------------------------------------
\documentclass[a4paper,11pt]{article}
% Layout
% ------------------------------------------------------------------------------
\input{relat_layout.tex}

%\usepackage{circuitikz}
\usepackage[clean]{svg}

\title{Projeto - Controle discreto de um motor DC} % Define o título do Relatório
\author{Rafael Lima}

% Definições Auxiliares ( Macros próprias )
% ------------------------------------------------------------------------------
%\input{relat_aux.tex} % Arquivo com minhas macros
% ----------------------------------~>ø<~---------------------------------------
\begin{document}
% Capa e Índice ----------------------------------------------------------------
\input{relat_capa.tex} % Capa para UnB
% Conteúdo ---------------------------------------------------------------------

\section{Objetivos}

\section{Desenvolvimento}

\section{Conclusão}

% ------------------------------------------------------------------------------
\newpage
% Referências
\addcontentsline{toc}{section}{Referências} % Adiciona linha no índice
\bibliographystyle{abbrv} % Define Estilo e gera bibliografia
\bibliography{references} % Adiciona Arquivo com Referências

% Acrescentadas no arquivo references.bib
% para usa-las no texto basta usar \citep{}
% para citar sem usar no texto basta usar \nocite{}
%\nocite{sympy}
%\nocite{pythontex}
%\nocite{matlabcontrol}
%\nocite{matlabsymbolic}
%\nocite{ogata2010modern}
%\nocite{nise2012}

% ------------------------------------------------------------------------------
\newpage
\section*{Anexos}
\addcontentsline{toc}{section}{Anexos} % Adiciona linha no indice
%\subsection*{Python}

%Para os cálculos e demonstrações foi utilizado o pacote \textit{Python}\TeX\ \cite{pythontex} para o \LaTeX\ em conjunto da bibliteca \textit{sympy}\cite{sympy}. Segue o script completo em python:

%\inputminted[xleftmargin=15pt,linenos,frame=single,framesep=5pt,breaklines=true]{python}{../python/exsim6.py}

%\newpage
\subsection*{Matlab}

%\subsubsection*{Parte 1}
%Para o desenho dos gráficos e simulações foi utilizado o \textit{Matlab} em conjunto das toolbox \textit{Control System}\cite{matlabcontrol} e \textit{Symbolic Math}\cite{matlabsymbolic}. Segue o código referente usado

%\inputminted[xleftmargin=15pt,linenos,frame=single,framesep=5pt,breaklines=true]{matlab}{../matlab/project.m}

% ------------------------------------------------------------------------------
\end{document}
